\documentclass[11pt]{article}
                                      
\usepackage{fullpage}					% full page dimensions
%\usepackage[letterpaper,hmargin=1in,vmargin=1in]{geometry}

\usepackage{amsmath}                    % special AMS math symbols
\usepackage{amssymb}                    % special AMS math symbols
\usepackage{color}                      % colored text and backgrounds

\usepackage{indentfirst}
\usepackage{framed}

\usepackage{algorithm} 
\usepackage{algpseudocode}

\usepackage{verbatim}

%\usepackage{graphicx}                   % graphics
%\usepackage{tabularx}

%\usepackage{epstopdf}
%\usepackage{amsthm}
%\usepackage{multirow}

%\usepackage{subcaption}
%\usepackage{comment}
%\usepackage{framed}
%\usepackage{hyperref}


\begin{document}

\title{Anonymous DTN routing}
\date{7/24/2013}
\maketitle

%%%%%%%%%%%%%%%%%%%%%%%%%%%%%%%%%%%%%%%%%%%%%%%%%%%%%%%%%%%%%%%%%%%%%%%%%%
\section{Introduction}
%%%%%%%%%%%%%%%%%%%%%%%%%%%%%%%%%%%%%%%%%%%%%%%%%%%%%%%%%%%%%%%%%%%%%%%%%%
In this project, we propose a routing protocol which provides \textit{identity anonymity} and \textit{location anonymity} to the participants in DTN.  Identity anonymity means that the identity of packet sender or destination is not revealed to any other nodes whom the sender or destination don't trust.  Location anonymity implies that the geographic location of a node cannot be tracked by other nodes whom the node doesn't trust.  

To achieve identity anonymity and location anonymity, we use the concept of \textit{ephemeral ID}.  A node generates an ephemeral ID periodically and the other nodes whom it trusts only can map the ephemeral ID to the permanent ID.  Then the node uses an ephemeral ID instead of its permanent ID to communicate with other nodes.  Since a packet delivered through the proposed routing protocol does not contain permanent IDs of sender and destination, other untrusted nodes cannot learn the permanent IDs of sender and destination from the packet.  Also, it is not possible to track a specific node since the ephemeral IDs of nodes are changed periodically. 


%%%%%%%%%%%%%%%%%%%%%%%%%%%%%%%%%%%%%
\subsection{Attack model}
%%%%%%%%%%%%%%%%%%%%%%%%%%%%%%%%%%%%%
\begin{enumerate}
\item Passive global eavesdropper.
\item Active attacker who is able to compromise a subset of untrusted nodes
\item Location tracking by learning permanent node ID through eavesdropping (passive attacker) or message handshake (active attacker)
\item No analog finger printing.
\end{enumerate}


%%%%%%%%%%%%%%%%%%%%%%%%%%%%%%%%%%%%%
\subsection{Problem Definition}
%%%%%%%%%%%%%%%%%%%%%%%%%%%%%%%%%%%%%
\label{problem}
\begin{enumerate}
\item Sender identity anonymity
  \begin{itemize}
  \item Only packet sender and destination learn the ephemeral ID and permanent ID of the packet sender. 
  \item Passive and active attackers learn nothing about the packet sender.
  \item Nodes that the sender trusts cannot learn anything about the packet sender. 
  \end{itemize}

\item Receiver identity anonymity against untrusted nodes
  \begin{itemize}
  \item Only packet sender, destination and trusted nodes of the packet destination learn the ephemeral ID and permanent ID of the packet destination. 
  \item Passive attackers learn nothing about the packet destination.
  \item Active attackers learn the ephemeral ID of the destination, but cannot map the ephemeral ID to the actual node/device.
  \end{itemize}

\item Unlinkability
  \begin{itemize}
  \item Passive or active attackers are not able to link any observed packets sent from a node to packets received by another node.   
  \end{itemize}

\item Location anonymity 
  \begin{itemize}
  \item Passive or active attackers cannot link an observed ephemeral address to the actual node/device.
  \end{itemize}

\item Reasonable efficiency (delivery rate, bandwidth, delivery latency)
	\begin{itemize}
	\item Compared to underlying routing protocol (TBD)
	\item Compared to other anonymous routing protocols (ARDEN, TOR)
	\end{itemize}

\end{enumerate}


%%%%%%%%%%%%%%%%%%%%%%%%%%%%%%%%%%%%%
\subsection{Assumption}
%%%%%%%%%%%%%%%%%%%%%%%%%%%%%%%%%%%%%
\begin{enumerate}
\item Each node has a group of trusted nodes and knows permanent IDs of the trusted nodes.
\item Loosely synchronized time between nodes.
\end{enumerate}

To make the problems simpler, we set a few more assumptions which may be removed or changed later. 

\begin{enumerate}
\setcounter{enumi}{2}
%\item A node has symmetric keys of other nodes.
\item A node knows random seeds and ephemeral IDs of its trusted nodes. 
\item Every node follow the locally identical epoch schedule.  Since the nodes are loosely synchronized, epoch schedule of each node is also loosely synchronized. 
\item Mutual trust: If Alice trusts Bob, Bob also trusts Alice. 
\item Anonymous routing is possible only when a packet sender trusts a packet destination. 
\end{enumerate}


%%%%%%%%%%%%%%%%%%%%%%%%%%%%%%%%%%%%%
\subsection{Project Goals}
%%%%%%%%%%%%%%%%%%%%%%%%%%%%%%%%%%%%%
In this project, we try to solve the problems stated in \ref{problem} by proposing a routing protocol which provides anonymity to nodes in DTN through the use of ephemeral ID.  In addition, the proposed protocol should show reasonable efficiency in terms of packet delivery rate, bandwidth and packet delivery latency.  

Our protocol achieves sender identity anonymity and weak receiver identity anonymity against untrusted nodes through the use of ephemeral ID.  
Each node $i$ in DTN generates and updates its own ephemeral ID, and other nodes that $i$ trusts are able to generate the current ephemeral ID of $i$ at any time.  
Given the permanent ID and the random seed number of $i$, the nodes that $i$ trusts can generate the ephemeral ID of $i$ at any time. 
However, attackers or nodes that $i$ does not trust cannot learn the permanent ID of $i$.  
Also the attackers may receive the current ephemeral ID of $i$ from $i$ or other nodes, but they cannot generate ephemeral ID of $i$ on their own.  
Therefore, even though the attackers observe or receive packets sent by $i$, they cannot determine the permanent ID of $i$.  
Moreover, the attackers recognize packets from a single node over different epochs as several streams of packets from several nodes, since the ephemeral ID of the sender is changed periodically. 

Unlinkability is also achieved by the use of ephemeral ID.  Unlinkability may be broken if packet delivery from a sender to a destination is completed within a single epoch, that is, ephemeral IDs of the sender and destination are not changed during the delivery.  Even in this case, attackers are only able to learn ephemeral IDs of the sender and destination and cannot map the ephemeral IDs to the permanent IDs.  Moreover, the length of epoch is set to be reasonably short so that ephemeral IDs of sender and destination contained in a packet would be changed at least once during the delivery. 

Location anonymity is achieved by the use of ephemeral ID which is changed periodically.  Since any kind of analog finger printing is not considered in this project, attackers cannot track a node without knowing a permanent ID and a random seed of a victim node that are used for generating ephemeral ID. 

Efficiency of the protocol is not fully considered in the current document. We try to avoid pure epidemic routing for bandwidth efficiency but for now, there's not much to say about packet delivery rate and packet delivery latency. 



%%%%%%%%%%%%%%%%%%%%%%%%%%%%%%%%%%%%%%%%%%%%%%%%%%%%%%%%%%%%%%%%%%%%%%%%%%
\section{Protocol}
%%%%%%%%%%%%%%%%%%%%%%%%%%%%%%%%%%%%%%%%%%%%%%%%%%%%%%%%%%%%%%%%%%%%%%%%%%

\subsection{Notation}
\begin{itemize}
\item $m$: Message to be sent. All messages are of the same size.
\item $c \leftarrow \{m\}_{k}$: Public (symmetric) key encryption of message $m$.
\item $m \leftarrow \{c\}_{k^{-1}}$: Public (symmetric) key decryption of $c$.
%\item $\{m\}_{k}$, $\{c\}_{k^{-1}}$: Public key encryption/decryption of message $m$.

\item $i$: A node in DTN. $i \in N$.
\item $id_i, eid_i$: Permanent ID and ephemeral ID of node $i$.
\item $r_i$: A random seed number of $i$.

\item $G_i$: A set of permanent IDs of trusted nodes of $i$.
\item $R_i$: A set of random seeds of nodes in $G_i$.
\item $E_i$: A set of ephemeral IDs of nodes in $G_i$. %It may include ephemeral IDs in used last $$-previous epochs. 
\item $H_i$: A set of ephemeral IDs of contact history (neighbor nodes) of $i$. $H_i$ contains ephemeral IDs that are not in $E_i$. %It may include ephemeral IDs in used last $m$-previous epochs. 
\item $BF_{E_i}$: Bloom filter encoding of ephemeral IDs of nodes in $G_i$.
\end{itemize}


\subsection{Network initialization}

\begin{framed}
\noindent
\textbf{Inputs:} A node $i$ has $G_i$ and $R_i$. \\

\noindent
\textbf{Outputs:} A node $i$ outputs $eid_i$ and $E_i$, a set of ephemeral IDs of the nodes in $G_i$.  $H_i$, the neighbor node list of $i$, is initialized. \\

\begin{algorithmic}[1]
  \Procedure{GenEphemeralID}{$id, r, t$}
    \State $eid \leftarrow \texttt{H}(id, r, t)$
    \State \textbf{return} $eid$
  \EndProcedure\\

  \Procedure{Initialize}{ }
    \State $eid_i \leftarrow \texttt{GenEphemeralID}(id_i, r_i, t)$

	\ForAll{$eid_j \in E_i$}
	  \State $E_i.eid_j \leftarrow \texttt{GenEphemeralID}(G_i.id_j, R_i.r_j, t)$
	\EndFor
	
	\State $H_i \leftarrow NULL$
    
    \State \textbf{return}  
  \EndProcedure  
\end{algorithmic}
\end{framed}

When a node $i$ enters a DTN network, it is assumed that $i$ is given $id_i$, $r_i$, $G_i$ and  $R_i$.  
Then $i$ is able to build $E_i$, a set of ephemeral IDs of nodes in $G_i$ by executing \texttt{Initialize()} procedure.  
On entering a DTN network or at the end of each epoch, a node $i$ executes \texttt{Initialize()} to generate new parameters $eid_i$ and $E_i$.  
$eid_i$, an ephemeral ID of $i$, is generated by \texttt{GenEphemeralID()} procedure where $eid_i$ is generated using a hash function $H()$ which takes the permanent id $id_i$, a random seed $r_i$ and time $t$ as inputs. (Line 7)  
Then $i$ updates $E_i$ by generating new ephemeral IDs of nodes in $G_i$. (Lines 8-10)  
At the end of each epoch, a node $i$ resets $H_i$ since it cannot generate new ephemeral IDs of untrusted nodes. (Line 11)

Ephemeral ID or changes of ephemeral ID should not reveal anything about a node.
An attacker should not be able to learn a permanent ID from an ephemeral ID.  
Also, an attacker should not be able to track a specific node by monitoring an ephemeral ID of the target node. 
In order to prevent such attacks, we apply the same epoch schedule to all nodes in DTN. 
Since nodes in DTN are assumed to be loosely synchronized to each other, an identical epoch schedule can be applied to all the nodes.  
That is, every node in DTN changes its own ephemeral ID almost at the same time.   
Therefore, it would not be possible for an attacker to identify a specific node by monitoring its unique timing of ephemeral ID changes.


\subsection{Beacon construction}
\begin{framed}
\noindent
\textbf{Inputs:} A node $i$ has $E_i$, a set of ephemeral IDs of the nodes in $G_i$, and $H_i$, a neighbor node list of $i$.

\noindent
\textbf{Outputs:} A node $i$ outputs a beacon message $beacon_i$.\\

\begin{algorithmic}[1]
  \Procedure{GenBeacon}{$id_i, E_i, H_i$}
    \State $beacon_i \leftarrow \{ eid_i \| BF_{E_i} \| BF_{H_i} \}$         
    \State \textbf{return} $beacon_i$
  \EndProcedure  
\end{algorithmic}
\end{framed}

After network initialization, $i$ advertises its presence to other nodes by generating a beacon message.  
The beacon message contains ephemeral ID of $i$, bloom filter encoding of $E_i$, and bloom filter encoding of $H_i$.  
$BF_{E_i}$ contains current ephemeral IDs of other nodes $i$ trusts, and $BF_{H_i}$ contains current ephemeral IDs of $i$'s neighbor nodes that $i$ does not trust.  
Ephemeral IDs of other nodes would be obtained through beacon messages from other nodes. 
%$H_i$ may include 1-hop neighbor nodes that $i$ has contacted directly, or may include any arbitrary $m$-hop neighbor nodes obtained by beacon messages from other nodes. 

Beacon message of $i$ should not reveal more than a current ephemeral ID of $i$. 
An attacker may try to learn ephemeral IDs contained in $BF_{E_i}$ or $BF_{H_i}$, possibly through brute force attack. 
However, an attacker cannot assure whether a certain ephemeral ID is in $BF_{E_i}$ because of the possibility of false positive match.  
Moreover, ephemeral IDs, $BF_{E_i}$ and $BF_{H_i}$ of nodes are changed at the end of an epoch that occurs to all nodes almost at the same time, an attacker cannot track a specific node on expiry of the current epoch. 

If beacon messages are transmitted exactly periodically, an attacker would be able to track a specific node by monitoring the timing of beacon message transmission. 
Therefore each node should put certain offset to vary its beacon period. 




\subsection{Packet construction}
\begin{framed}
\noindent
\textbf{Inputs:} A sender $n_s$ knows the ephemeral ID of the destination $n_d$ and shares a symmetric key $k_{sd}$ with $n_d$. \\

\noindent
\textbf{Outputs:} $n_s$ outputs a packet $p$. \\

\begin{algorithmic}[1]
  \Procedure{GenPacket}{$m, k_{sd}, eid_s, eid_d$}
    \State $c \leftarrow \{m\}_{k_{sd}}$
    \State $p \leftarrow \{eid_s \| eid_d \| c\}$
    \State \textbf{return} $p$
  \EndProcedure
\end{algorithmic}
\end{framed}

A sender node $n_s$ constructs a packet by first encrypting the message using a symmetric key $k_{sd}$ shared between the sender and destination of the packet.  
Then the sender generate $p$ which contains the encrypted message along with the ephemeral ID of the sender and destination. 

The packet can be accessed by all relay nodes on the route from the source to the destination, regardless of whether the relay nodes are trusted by the source node or not.
Therefore, we need to assure that any untrusted relay node cannot learn anything from the packet, and any trusted relay node cannot learn anything about the message $m$ included in the packet.
Since the message $m$ is encrypted using a symmetric shared key $k_{sd}$, any relay node without the key $k_{sd}$ cannot learn the message $m$.
The only things that a relay node can learn are ephemeral IDs of the source and destination nodes. 
However, untrusted relay node cannot learn the permanent IDs of the source or destination nodes from their ephemeral IDs. 



\subsection{Packet forwarding}
\begin{framed}
\noindent
\textbf{Inputs:} An intermediate node $i$ has a packet $p$ to forward. $i$ and $n_j$ are on contact.  $i$ has $beacon_j$, a beacon message of $n_j$, and a symmetric key $k_{ij}$.	\\

\noindent
\textbf{Outputs:} $i$ either transmits an encrypted packet $p'$ to $n_j$ or remove $p$ from the queue.\\

\begin{algorithmic}[1]
  \Procedure{ForwardPacket}{$p, beacon_j$}
    \If{$(trusted\_tx\_count[packet] + untrusted\_tx\_count[packet]) \ge TH_{total}$}
      \State \texttt{RemoveFromQueue}$(packet)$
  	  \State \textbf{return}
    \EndIf	\\


    \If{$epoch\_expired$}      
      \If{$p.eid_s \in E_i$}
      \Comment{$i$ is trusted by $n_s$ ($n_s \in G_i \leftrightarrow n_i \in G_s$).}            
        \State $id_s \leftarrow \texttt{MatchID}(packet.eid_s)$      
        \State $p.eid_s \leftarrow \texttt{GenEphemeralID}(id_s, r_s, t) $
      \EndIf     

      \If{$p.eid_d \in E_i$}
      \Comment{$i$ is trusted by $n_d$ ($n_d \in G_i \leftrightarrow n_i \in G_d$).}            
        \State $id_d \leftarrow \texttt{MatchID}(packet.eid_d)$      
        \State $p.eid_d \leftarrow \texttt{GenEphemeralID}(id_d, r_d, t) $
      \EndIf     
    \EndIf	\\
    
	\If{$p.eid_d = beacon_j.eid_j$}
	\Comment{$n_j$ is the destination of $p$.}
      \State $p' \leftarrow \{packet\}_{k_{ij}}$
      \State \texttt{SendPacket}$(packet, beacon_j.eid_j)$
      \State \texttt{RemoveFromQueue}$(packet)$
      \State \textbf{return}

	\ElsIf{$p.eid_s \in beacon_j.BF_{E_j}$}
    \Comment{$n_j$ is trusted by $n_s$($n_s \in G_j \leftrightarrow n_j \in G_s$).}            
      \State $p' \leftarrow \{packet\}_{k_{ij}}$
      \State $trusted\_tx\_count[packet]++$
      \State \texttt{SendPacket}$(packet, beacon_j.eid_j)$
      \State \textbf{return}

    \ElsIf{$untrusted\_tx\_count[packet] < TH_{untrusted}$}     
      \If{$p.eid_d \in beacon_j.BF_{H_j}$}
      \Comment{$n_j$ is not trusted by $n_s$, but $n_d$ is in the neighbor list of $n_j$}      
        \State $p' \leftarrow \{packet\}_{k_{ij}}$
        \State $untrusted\_tx\_count[packet]++$
        \State \texttt{SendPacket}$(p, beacon_j.eid_j)$
        \State \textbf{return}
      \EndIf
    \EndIf
    
    \State \textbf{return}
  \EndProcedure	\\
  
  \Procedure{ValidateBeacon}{$beacon_j$}
	\If{$eid_i \in beacon_j.BF_{E_j}$ AND $beacon_j.eid_j \notin E_i$}
	\Comment{$i$ does not trust $n_j$ but $eid_i$ is included in $BF_{E_j}$.}
	  \State \textbf{return}$(False)$
	\Else
	  \State \textbf{return}$(True)$
	\EndIf
  \EndProcedure

\end{algorithmic}

\end{framed}

A node $i$ has a packet to relay, or has generated a packet and wants to send to destination node.  Then whenever $i$ contacts $n_j$, that is, $i$ receives a beacon message from $n_j$, $i$ performs a procedure \texttt{ForwardPacket()} to forward the packet. 

We put a certain threshold for a packet replication to avoid pure epidemic routing. (Lines 2-5)
Then $i$ checks if the current epoch has been expired and if it has, $i$ tries to updates ephemeral IDs of sender and destination contained in the packet.  $i$ is only able to update $eid_s$ (or $eid_d$) only if $i$ is trusted by $n_s$ (or $n_d$). (Lines 7-16)
If $n_j$ is the destination node of the packet, $i$ forwards the packet and removes the packet from its queue.  However, $n_j$ may not remove the packet from the queue for receiver anonymity from eavesdroppers. (Lines 18-22)  If $n_j$ is not the destination but trusted by $n_s$, then $n_j$ forwards the packet to $n_j$ and keeps the packet for further replication. (Lines 23-27)  If $n_j$ is not the destination nor a node trusted by $n_s$, $n_j$ checks if $n_j$ has an ephemeral ID of the destination $eid_d$ as its neighbor node in $BF_{H_j}$.  $n_j$ forwards the packet to $n_j$ only if $eid_d$ is in $BF_{H_j}$ to prevent possible bandwidth waste. (Lines 28-35)


The forwarding protocol should work properly even if a few untrusted nodes are compromised and try to receive packets from packets from a relay node, possibly by using a fabricated beacon message. 
A compromised node $n_j$ may try to fabricate its $E_j$ so that a relay node $i$ considers $n_j$ as a node trusted by the source node $n_s$. 
An example of the attack would be setting $E_j$ to all $1$s so that the compromised node $n_j$ can be considered to be trusted by any other nodes. 
This attack would waste replication threshold $TH_{total}$ of $i$ and may deter packet forwarding to a proper route. 
To prevent this attack, the relay node $i$ may confirm if its own ephemeral ID $eid_i$ is included in $E_j$. 
If $n_j$ is not a trusted node of the relay node but $eid_i$ is included in $E_j$, $i$ may refuse to forward the packet to $n_j$. (Lines 39-45)



%%%%%%%%%%%%%%%%%%%%%%%%%%%%%%%%%%%%%%%%%%%%%%%%%%%%%%%%%%%%%%%%%%%%%%%%%%
\section{Simulator}
%%%%%%%%%%%%%%%%%%%%%%%%%%%%%%%%%%%%%%%%%%%%%%%%%%%%%%%%%%%%%%%%%%%%%%%%%%
\subsection{DTN Code for ns-2 and ns-3}
\begin{itemize}
\item \begin{verbatim} https://wiki.aalto.fi/display/~jlakkako@aalto.fi/DTN+Code+for+ns-2+and+ns-3} \end{verbatim}
\item The latest version has been released recently in 2012.
\item Supported DTN routing protocols
  \begin{itemize}
  \item Epidemic routing
  \item Binary spray and wait
  \end{itemize}

\item Supported network model
  \begin{itemize}
  \item Random walk
  \item SF cab trace or Helsinki city center trace (from ONE simulation)
  \end{itemize}

\item Other functionality
  \begin{itemize}
  \item Return receipts / antipackets
  \item A simple congestion control algorithm
  \end{itemize}  
\end{itemize}


\subsection{DTNSim2}
\begin{itemize}
\item \begin{verbatim} http://watwire.uwaterloo.ca/DTN/sim/ \end{verbatim}
\item The latest version has been released in 2006.
\item Used in ARDEN paper.  Seems like an official simulator introduced in Delay Tolerant Networking Research 
\item NO documentation at all.
Group.
\item Supported DTN routing protocols
  \begin{itemize}
  \item Epidemic routing (MEED)
  \item Unicast routing (ED)
  \end{itemize}

\item Supported network model
  \begin{itemize}
  \item ARDEN uses Haggle and RollerNet dataset
  \end{itemize}
\end{itemize}


\subsection{The ONE simulator}
\begin{itemize}
\item \begin{verbatim} http://www.netlab.tkk.fi/tutkimus/dtn/theone/ \end{verbatim}
\item The latest version has been released in 2011.
\item Supported DTN routing protocols
  \begin{itemize}
  \item Several epidemic routing protocols
  \item Spray and Focus (variant of spray and wait)
  \item Prophet
  \end{itemize}

\item Supported network model
  \begin{itemize}
  \item Random walk
  \item Helsinki city center trace
  \item Working Day Movement model
  \end{itemize}

\end{itemize}








\end{document}







