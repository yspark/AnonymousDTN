\documentclass[11pt]{article}
                                      
\usepackage{fullpage}					% full page dimensions
%\usepackage[letterpaper,hmargin=1in,vmargin=1in]{geometry}

\usepackage{amsmath}                    % special AMS math symbols
\usepackage{amssymb}                    % special AMS math symbols
\usepackage{color}                      % colored text and backgrounds

\usepackage{indentfirst}
\usepackage{framed}

\usepackage{algorithm} 
\usepackage{algpseudocode}

\usepackage{verbatim}

\usepackage{graphicx}                   % graphics
\usepackage[caption=false, font=footnotesize]{subfig}

\usepackage{multirow}

%\usepackage{tabularx}

%\usepackage{epstopdf}
%\usepackage{amsthm}
%\usepackage{multirow}

%\usepackage{subcaption}
%\usepackage{comment}
%\usepackage{framed}
%\usepackage{hyperref}


\begin{document}

\title{Anonymous DTN routing}
\maketitle

%%%%%%%%%%%%%%%%%%%%%%%%%%%%%%%%%%%%%%%%%%%%%%%%%%%%%%%%%%%%%%%%%%%%%%%%%%
\section{Experimental Result}
%%%%%%%%%%%%%%%%%%%%%%%%%%%%%%%%%%%%%%%%%%%%%%%%%%%%%%%%%%%%%%%%%%%%%%%%%%
\subsection{Overview}

\subsubsection{Simulation model}
\begin{itemize}
 \item ONE simulator, modified default scenario/setting

 \item Map: Helsinki (4500m * 3500m)

 \item Simulation running time: 12 hours

 \item Nodes: 246 (160 humans, 80 cars, 6 trams)
  \begin{itemize}
   \item Packet buffer: Humans and cars (50MB), trams (500MB).
   \item Contact interval: Humans (2 mins 30 secs), cars (1 min), trams ( 40 secs)
  \end{itemize}

 \item Packet(message) generation
  \begin{itemize}
   \item Packet size: 500KB - 1MB
   \item Packet generation interval: 35sec - 50sec
   \item TTL: 5 hours
   \item Packet generation stopped when 5 hours (packet TTL) are left.   
   \item Total number of packets generated: about 575
  \end{itemize}

 \item Movement: Random way point, map-based movement.

 \item Network interface: bluetooth, wlan (determine communication distance and bandwidth)
  \begin{itemize}
   \item Humans, cars: Bluetooth (Bandwidth: 2Mbps, Communication range: 10m)
   \item Trams: WLAN (Bandwidth: 10Mbps, Communication range: 100m)
  \end{itemize}

\end{itemize}



\subsubsection{Anonymous DTN routing setup}
\begin{itemize}
 \item \# group: 1
 \item \# nodes in a group: [5\%, 10\%, 15\%, 20\%, 25\%]
 \item Epoch: [10mins, 20mins, 30mins, 60mins]
 \item Ephemeral ID duration: [3 epochs, 6 epochs]
 \item Base routing protocol: epidemic (flooding)
\end{itemize}



\subsubsection{Assumptions \& simplification}
\begin{itemize}
% \item Communication within a group\\
%Only nodes belong to any ``group'' can send packets to other nodes it trusts. 
%Nodes that don't belong to any group cannot generate packets.

 \item Strict time sync\\
Epoch starts exactly at the same time in all nodes

 \item No ``beacon'', ``hello'', ``pull'' messages\\
 Once two nodes are located within a specific distance, they know ephemeral addresses, packet digest, pulling list of each other without any message exchange. 

 \item ``Out-of-group'' nodes do not use ephemeral IDs.\\
 Those nodes use permanent IDs which are not changed during the simulation.

 \item Forwarding policy\\
On contact, a node first forwards packets whose destinations are either trusted by the next-hop node or in neighbor list of the next-hop node.  Then it tries to forward remaining packets in FIFO manner. 
\end{itemize}







\clearpage
\subsection{Results}


%\clearpage
\subsubsection{Communication among all nodes}
In this test scenario, every node can send packets to any other nodes.
Packet generation follows rules below:
\begin{itemize}
\item Nodes belong to the group generate and receive about 20\% of overall packets generated during the simulation. 
\item For the rest 80\% of packet generation, sender and receiver are randomly selected from all node. 
\end{itemize}



\begin{table}[!h]
\center
\begin{tabular}[!h]{|c|c|c|}
\hline
Ephemeral ID duration	& Trusted nodes \%	& Epoch	\\	
\hline
\hline
\multirow{2}{*}{3 epochs}	& 5\%				& 60 mins	\\
							& 10\%				& 30 mins	\\
\hline
\multirow{2}{*}{6 epochs}	& 5\%			& 30 mins \\
							& 10\%			& 20 mins \\
\hline
\end{tabular}
\vspace{10pt}
\caption{{ \bf Example settings with overall delivery rate of about 90\% (Flooding: 92.91\%).}}
\label{tab:dataset_summary}
\end{table}




% delivery rate
\begin{figure}[h!]
\center
\subfloat[Ephemeral ID valid for 3 epochs]{%
\includegraphics[width=0.49\columnwidth]{figures/epoch_3_overall/delivery_rate.pdf}
\label{fig:delivery_rate_3_overall}
}
\hfill
\subfloat[Ephemeral ID valid for 6 epochs]{%
\includegraphics[width=0.49\columnwidth]{figures/epoch_6_overall/delivery_rate.pdf}
\label{fig:delivery_rate_6_overall}
}

\caption{{\bf Overall packet delivery rate.} 
Delivery rate of pure epidemic routing: 92.91\%.  
With epoch=60 mins, overall delivery rate is almost similar to that of epidemic routing regardless of the percentage of trusted nodes. 
In Figure~\ref{fig:delivery_rate_6_overall}, delivery rates with epoch $\ge$ 20 mins are about 90\%, especially when the percentage of trusted nodes $\ge 10\%$.
}
\label{fig:delivery_rate}
\end{figure}



% delivery rate retails
\begin{figure}[h!]
\center
\subfloat[In-group to In-group. Ephemeral ID valid for 3 epochs]{%
\includegraphics[width=0.33\columnwidth]{figures/epoch_3_overall/delivery_rate_t_to_t.pdf}
\label{fig:delivery_rate_3_t_to_t}
}
\subfloat[In-group to In-group. Ephemeral ID valid for 6 epochs]{%
\includegraphics[width=0.33\columnwidth]{figures/epoch_6_overall/delivery_rate_t_to_t.pdf}
\label{fig:delivery_rate_6_t_to_t}
}
\hfill

\subfloat[In-group to Out-of-group. Ephemeral ID valid for 3 epochs]{%
\includegraphics[width=0.33\columnwidth]{figures/epoch_3_overall/delivery_rate_t_to_ut.pdf}
\label{fig:delivery_rate_3_t_to_ut}
}
\subfloat[In-group to Out-of-group. Ephemeral ID valid for 6 epochs]{%
\includegraphics[width=0.33\columnwidth]{figures/epoch_6_overall/delivery_rate_t_to_ut.pdf}
\label{fig:delivery_rate_6_t_to_ut}
}
\hfill

\subfloat[Out-of-group to In-group. Ephemeral ID valid for 3 epochs]{%
\includegraphics[width=0.33\columnwidth]{figures/epoch_3_overall/delivery_rate_ut_to_t.pdf}
\label{fig:delivery_rate_3_ut_to_t}
}
\subfloat[Out-of-group to In-group. Ephemeral ID valid for 6 epochs]{%
\includegraphics[width=0.33\columnwidth]{figures/epoch_6_overall/delivery_rate_ut_to_t.pdf}
\label{fig:delivery_rate_6_ut_to_t}
}
\hfill

\subfloat[Out-of-group to Out-of-group. Ephemeral ID valid for 3 epochs]{%
\includegraphics[width=0.33\columnwidth]{figures/epoch_3_overall/delivery_rate_ut_to_ut.pdf}
\label{fig:delivery_rate_3_ut_to_ut}
}
\subfloat[Out-group to Out-group. Ephemeral ID valid for 6 epochs]{%
\includegraphics[width=0.33\columnwidth]{figures/epoch_6_overall/delivery_rate_ut_to_ut.pdf}
\label{fig:delivery_rate_6_ut_to_ut}
}

\caption{{\bf Detailed packet delivery rate.} 
Delivery rate of packets destined for `out-of-group' is almost same to that of epidemic routing, regardless of epoch and percentage of trusted nodes. 
Delivery rate of `Out-of-group' to `In-group' is relatively low, ranging from 20\% to 80\%. 
}
\label{fig:delivery_rate}
\end{figure}





% delivery latency
\begin{figure}[h!]
\center
\subfloat[Ephemeral ID valid for 3 epoch]{%
\includegraphics[width=0.49\columnwidth]{figures/epoch_3_overall/delivery_latency.pdf}
\label{fig:delivery_latency_3_overall}
}
\hfill
\subfloat[Ephemeral ID valid for 6 epochs]{%
\includegraphics[width=0.49\columnwidth]{figures/epoch_6_overall/delivery_latency.pdf}
\label{fig:delivery_latency_6_overall}
}

\caption{{\bf Overall packet delivery latency.}
Overall packet delivery rate is almost same to that of epidemic routing. 
}
\label{fig:delivery_latency_overall}
\end{figure}




% hop count
\begin{figure}[h!]
\center
\subfloat[Ephemeral ID valid for 3 epochs]{%
\includegraphics[width=0.49\columnwidth]{figures/epoch_3_overall/hopcount.pdf}
\label{fig:delivery_hopcount_3_overall}
}
\hfill
\subfloat[Ephemeral ID valid for 6 epochs]{%
\includegraphics[width=0.49\columnwidth]{figures/epoch_6_overall/hopcount.pdf}
\label{fig:delivery_hopcount_6_overall}
}

\caption{{\bf Overall packet delivery hop count.}
Overall delivery hop count is almost same to that of epidemic routing. 
}
\label{fig:delivery_hopcount_overall}
\end{figure}



\end{document}







