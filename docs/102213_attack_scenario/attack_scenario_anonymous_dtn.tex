\documentclass[11pt]{article}
                                      
\usepackage{fullpage}					% full page dimensions
%\usepackage[letterpaper,hmargin=1in,vmargin=1in]{geometry}

\usepackage{amsmath}                    % special AMS math symbols
\usepackage{amssymb}                    % special AMS math symbols
\usepackage{color}                      % colored text and backgrounds

%\usepackage{indentfirst}
\usepackage{framed}

\usepackage{algorithm} 
\usepackage{algpseudocode}

\usepackage{verbatim}

%\usepackage{graphicx}                   % graphics
%\usepackage{tabularx}

%\usepackage{epstopdf}
%\usepackage{amsthm}
%\usepackage{multirow}

%\usepackage{subcaption}
%\usepackage{comment}
%\usepackage{framed}
%\usepackage{hyperref}


\begin{document}

\title{Anonymous DTN routing: Attack scenario}
\maketitle




%%%%%%%%%%%%%%%%%%%%%%%%%%%%%%%%%%%%%%%%%%%%%%%%%%%%%%%%%%%%%%%%%%%%%%%%%%
\section{Attack model}
%%%%%%%%%%%%%%%%%%%%%%%%%%%%%%%%%%%%%%%%%%%%%%%%%%%%%%%%%%%%%%%%%%%%%%%%%%
In our attack scenario, an adversary tries to track a victim node or infer trusted nodes of a victim node by injecting a ``marker packet'' and later confirming that the victim node has the marker packet.  
An adversary can confirm the marker packet through 1) destination address of the marker packet, 2) packet ID of the marker packet and 3) payload of the marker packet.


\subsection{Notation}
\begin{itemize}
\item $v$: victim node
\item $a$: adversary node
\item $T(v)$: group of nodes that $v$ trusts
\item $D(v)$: list of destinations of packets buffered in $v$
\item $P_{d}$: packet destined for node $d$.
\end{itemize}



\subsection{Tracking a victim node}
\begin{enumerate}
\item $a \rightarrow v: P_{a'}, a' \notin T(v)$	\\
An adversary $a$ sends a marker packet $P_{a'}$ to a victim $v$.

\item Epoch change: $a \Rightarrow a'$, $v \Rightarrow v'$ 

\item $a' \leftarrow v: P_{a'}$	\\
The adversary $a'$ pulls the marker packet from $v'$ and determines that $v'$ is $v$.
\end{enumerate}




\subsection{Inferring victim's trusted nodes}
\begin{enumerate}
\item $T(v) = \{m, n, o, p\}$	\\
$B(v) = \{m, n, x, y\}$		\\
$B(a) = \{p, s, t\}$	\\
Victim $v$ has its trusted node $\{m, n, o, p\}$ and buffered packets destined for $\{m, n, x, y\}$.
Adversary $a$ prepares marker packets destined for $\{p, s, t\}$.

\item $a \rightarrow v: P_{p}$	\\
$v$ pulls packet $P_p$ from $a$, possibly using PIR so that $a$ cannot know which packets are pulled by $v$. 


\item $a \Rightarrow a'$	\\
$B(v) = \{m, n, p, x, y\}$	\\
$a$ changes its ephemeral ID to $a'$ and gets $B(v)$ that $p$ is newly added to. 
Then $a$ determines that $v$ pulled packets destined for $p$ and therefore, $v$ trusts $p$. 


\end{enumerate}


%%%%%%%%%%%%%%%%%%%%%%%%%%%%%%%%%%%%%%%%%%%%%%%%%%%%%%%%%%%%%%%%%%%%%%%%%%
\section{Secure protocol}
%%%%%%%%%%%%%%%%%%%%%%%%%%%%%%%%%%%%%%%%%%%%%%%%%%%%%%%%%%%%%%%%%%%%%%%%%%
\subsection{Notation}
\begin{itemize}
\item $\rightarrow$: packet transmission between two untrusted nodes
\item $\Rightarrow$: packet transmission between two trusted nodes
\end{itemize}


\subsection{Packet forwarding scenarios}

\subsubsection{Packet destined for a untrusted node}
Group of trusted nodes $T = \{a, b, c, d\}$	\\
Untrusted nodes (potential adversaries) $U = \{x, y, z\}$	\\
Packet $P_z$: packet destined for $z \notin T$	\\


\begin{enumerate}
\item $a \Rightarrow b \Rightarrow ... \Rightarrow d$: ok

\item $x \rightarrow a \Rightarrow b \Rightarrow ... \Rightarrow d$: ok

\item $x \rightarrow a	\rightarrow y$: ok	\\
It reveals that node $a$ pulled the packet $P_z$ from $x$, but it does not reveal any secret of $a$ (e.g., two or more ephemeral IDs of node $a$ or trusted nodes of $a$)


\item $x \rightarrow a$ (epoch change) $a' \rightarrow y$: controversial \\
Since $a$ cannot change the packet destination and payload, the adversary $y$ is able to infer that $a'$ is $a$. 
However, the adversary cannot be sure that $a$ and $a'$ are the same node since 1) $a$ might have sent $P_z$ to other nodes and 2) lists of packet destinations buffered in $a$ and $a'$ are different due to the epoch change. 


\item $x \rightarrow a \Rightarrow b \rightarrow y$: controversial	\\
The adversary is able to infer that $b$ received $P_z$ from $a$ and therefore, $b$ trusts $a$. 
However, more than one intermediate nodes can exist between $a$ and $b$ and in this case, $b$ may not trust $a$. 

\end{enumerate}



\subsubsection{Packet destined for a trusted node}
Group of trusted nodes $T = \{a, b, c, d\}$	\\
Untrusted nodes (potential adversaries) $U = \{x, y, z\}$	\\
Packet $P_d$: packet destined for $d \in T$	\\



\begin{enumerate}
\item $a \Rightarrow b \Rightarrow ... \Rightarrow d$: ok

\item $x \rightarrow a \Rightarrow b \Rightarrow ... \Rightarrow d$: ok

\item $x \rightarrow a	\rightarrow y$: controversial	\\
It may reveal that $a$ pulled packet destined for $d$ and therefore, $a$ trusts $d$. 
To prevent the leakage,
 \begin{enumerate}
 \item $a$ changes packet destination $d$ to $d'$.  One naive way is to append a random number $k$ and encrypt the destination address using the symmetric key shared among the trusted nodes.
 \item $a$ changes packet ID
 \item $a$ encrypts the payload of $P_d$ (e.g., encryption using symmetric key shared among trusted nodes) 
 \end{enumerate}


\item $x \rightarrow a$ (epoch change) $a' \rightarrow y$: ok \\
$a$ needs to change packet ID and encrypt the payload, just as in previous scenario.

\item $x \rightarrow a \Rightarrow b \rightarrow y$: ok	\\




\end{enumerate}





















\end{document}







